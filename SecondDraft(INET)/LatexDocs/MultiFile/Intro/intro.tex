% ---------------------------------------------------------------- INTRODUCTION -----------------------------------------------------------------------%

Internet connectivity has become ubiquitous in the modern society. Not only individual users but a multitude of critical services and business enterprises depend on a reliable network connection. Hence, over the years, a lot of resources and studies have been dedicated towards providing a more reliable service \hyperlink {K4}{[4 - 6]}. Among many other aspects, service availability is the most important measure of reliability. And judging the reliability of a network based on its availability should first involve understanding the cause, duration and frequency of failures across the network.
  
The authors in this paper present a prediction model to analyze and breakdown causes of internet disruptions and outages by studying the data from a large CDN and thereby try to provide a fine grain understanding of when and why these disruptions occur. 

Their study is based on the CDN access logs which contain entries of per hour requests sent by the connected address blocks. A sudden absence of entry in the log indicates a loss of internet connectivity in the specific IP address blocks, this is called a disruption. A service outage on the other hand means loss of internet access service in the address blocks. Therefore, although a disruption might be caused due to an actual service outage, every disruption does not necessarily mean that there was a service outage but instead a disruption might mean that the public IP of the host has changed or has been reassigned. Using the CDN log data as well as connectivity log from individual devices, collected over a selected duration, authors present a detailed study featuring size, timing and frequency of disruptions to argue that large share of detected disruptions or service outages might be due to planned human activities such as server maintenance and user mobility and thus might not always indicate network failures or issues. Another important aspect of the paper is that the authors, through the data, make a distinction between which disruptions represent an actual service outage and which are anti-disruptive events and not an actual service unavailability. Further, they compare their findings with Trinocular \hyperlink {K2}{[2]}, an established tool to measure internet outage to validate their results.

The rest of the paper is organized as follows: We discuss the related works in \hyperref[RW]{Section 2}. \hyperref[M]{Section 3} details the implemented methodology. \hyperref[RandEv]{Section 4} analyzes and interprets the results and also presents a comparison of results with Trinocular and finally \hyperref[DC]{Section 5} provides an overview of the work and discusses the implications of the work.



