There are a lot of research works in internet outage detection and reliability. Turner et al. \hyperlink {K3} {[3]} use router configuration files, syslog archives and operational mailing list to recreate failure events in a network. They then perform an extensive analysis of five years of such data from a network consisting of over two hundred routers. Basically, they extract the failure events from the syslog and router configuration files over the defined period and deduce cause of failure by referring to the administration email logs. The data is then analyzed for failure duration, cause and impact. One of the most significant features of their work is that the data sources used is easily available and is not restricted to a research group or require any special permission. Similarly, Banerjee \hyperlink {K1} {[1]} et al. use the outage mailing list \hyperlink{K17}{[17]} to investigate outages. The outage mailing list contains outage reports and discussion, impacts and troubleshooting information relating to major outages. This information is parsed using NLP and are then clustered into labeled categories using machine learning techniques and are then analyzed to derive probable cause, impact and type of disruption. The experiment was performed on outage data accumulated over seven years.

Benson et al. \hyperlink {K7} {[7]} use IBR traffic analysis to detect disruptions. Their methodology is rather interesting. Their system is based on observing IBR traffic from passive darknet \hyperlink {K8} {[8]}. IBR basically constitutes of TCP SYNs trying to connect to hosts but since darknet does not respond, these signals have to be retransmitted. These retransmissions follow a consistent pattern and thus the absence of these retransmission signals can be considered as packet loss and can be used to detect disruptions.

Trinocular \hyperlink {K2} {[2]} an outage detection system, uses active probing to detect disruptions. It sends multiple ICMP probes per hour to over 3 million /24 address blocks. If it detects any unresponsive prefixes, it flags it as a disruption. The authors compare their system against the disruptions detected by Trinocular.
